\documentclass{article}
\usepackage{amsmath,amssymb,stmaryrd}
\def\MM#1{\boldsymbol{#1}}
\newcommand{\pp}[2]{\frac{\partial #1}{\partial #2}} 
\newcommand{\dede}[2]{\frac{\delta #1}{\delta #2}}
\newcommand{\dd}[2]{\frac{\diff#1}{\diff#2}}
\newcommand{\dt}[1]{\diff\!#1}
\DeclareMathOperator{\Hess}{Hess}
\def\MM#1{\boldsymbol{#1}}
\DeclareMathOperator{\diff}{d}
\DeclareMathOperator{\DIV}{DIV}
\DeclareMathOperator{\D}{D}
\bibliographystyle{plain}
\newcommand{\vecx}[1]{\MM{#1}}
\newtheorem{definition}{Definition}
\newcommand{\code}[1]{{\ttfamily #1}} 
%uncomment \solnsfalse to remove solution set
\newif\ifsolns
%\solnstrue
\solnsfalse

\usepackage{fancybox}
\begin{document}
\title{Equation notes}
\author{all of us}
\maketitle

\begin{align}
  u_t + u\cdot\nabla u + fu^{\perp} + g\nabla (D-H_0+b) = 0, \\
  D_t + \nabla\cdot (uD) = 0.
\end{align}
$\eta$ version.

\begin{align}
  u_t + u\cdot\nabla u + fu^{\perp} + g\nabla \eta = 0, \\
  \eta_t + \nabla\cdot (u(H_0-b+\eta)) = 0.
\end{align}

Splitting (without advection)

\begin{align}
  u_t &= \underbrace{- fu^{\perp} - g\nabla \eta}_{L_u} -u\cdot\nabla u, \\
  \eta_t &=\underbrace{- \nabla\cdot (u(H_0-b))}_{L_\eta} -\nabla\cdot (u\eta).
\end{align}


\end{document}

